\documentclass{article} %文章模板
\usepackage{ctex} %导入中文宏包
%\parindent=0pt %在导言区设置全文缩进的距离
\title{训练} %标题
\author{刘圻浩 \and LiuQihao\thanks{作者附加信息}} %作者和附加信息
\date{\today} %日期



\begin{document}
\maketitle %把标题显示出来

\% %显示百分号%
\textbackslash %显示斜杠\
\{\} %显示{}
\# %显示#

这是第一段第一行\\
这是第二行\\
这是第三行\\
第四行\\

这是第二段第一行\\ %上面的空行可以分段
第二行\\
第三行\\
第四行\par %分段
第三段第一行\\
这些符号的大小\hspace{1em}写法\\ %水平空格,1em是一个中 文字符大小,em是单位
\begin{center} %里面的文字会直接分段
    这是居中的文字
\end{center}
\begin{flushleft}
    这是水平居左
\end{flushleft}\par
\begin{center}
冬天的美丽   
\end{center}\par
冬天是一个美丽而神秘的季节。尽管寒冷,但它带来了独特的景色和体验。
\begin{flushright}
冬天最令人期待的景象莫过于雪景。
\end{flushright}\par
%\noindent 接下来的文字没有缩进
洁白的雪花纷纷扬扬地从天空飘落,覆盖了大地,仿佛给世界披上了一层银装。孩子们在雪地里堆雪人、打雪仗,欢声笑语回荡在空中。\par
在寒冷的冬天,家是最温暖的避风港。家人们围坐在火炉旁,喝着热腾腾的茶,聊着天,享受着温馨的时光。窗外寒风凛冽,屋内却温暖如春。
冬天还有许多美食让人垂涎欲滴。热气腾腾的火锅、香喷喷的烤红薯、暖心的汤圆,这些美食不仅温暖了身体,也温暖了心灵。\par\noindent
冬天虽然寒冷,但它带来了无尽的美丽和温暖。让我们一起享受这个美丽的季节吧!

\end{document}